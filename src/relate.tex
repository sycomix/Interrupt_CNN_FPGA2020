\label{sec:relate}
\subsection{ CNN-based FE and PR }

Previous feature-point extraction works usually consist of two parts: 1) feature-point detection and 2) descriptors generation.
Mur-Artal \cite{Mur-Artal:2017281} proposes a popular open-source SLAM system, ORB-SLAM. ORB-SLAM uses the oFAST (OrientedFeature from Accelerated Segment Test \cite{biadgie2014feature}) detector to locate the feature-points and the BRIEF (Binary Robust Independent ElementaryFeatures \cite{calonder2010brief}) to generate the descriptor for each feature-point in binary strings. 
Simo-Serra \cite{simo2015discriminative} proposes a CNN-based descriptor generator that does not perform any feature-point detection. 
DeTone \cite{detone2018superpoint} presents a fully CNN-based feature-point extraction method that implements feature-point detection and descriptors generation using one CNN network.

Before CNN was introduced into place recognition, BoW \cite{small_1} models relying on handcrafted representation attract most of the attention. The accuracy of BoW-based methods is strongly influenced by the size of codebooks. Larger codebooks(~1M) \cite{large_1, large_2} can compete with CNN-based methods in accuracy, but they take up huge storage and communication resources. Smaller codebooks\cite{small_1, small_2} require less space but get worse results. In contrast to traditional methods, CNN-based methods not only perform well but also generate compact features. GeM \cite{radenovic2018fine} outperforms other state-of-art CNN-based place recognition methods like NetVLAD \cite{arandjelovic2016netvlad} and is used in this work.

\subsection{ FPGA accelerators for robot }

The feature-point extraction (FE) operation is the basic element of a visual-based robot.
Some previous works design hardware architectures for FE.
SRI-SURF \cite{jia2016sri} optimizes the memory access to speed-up SURF \cite{bay2006surf} feature-point extraction method. Recent years, ORB is widely used to extract feature-point for its speed and accuracy. 
Fang \cite{fang2017fpga} directly implements ORB on FPGA using HLS. Liu \cite{liu2019eslam} optimizes the ORB algrithm and designs a hardware for better performance.
Some other works design architectures for the entire robot system. Hero \cite{shi2018hero} is a framework for navigation and laser-based SLAM and cannot support visual-based SLAM, which is much more lightweighted and cheaper. These specific architectures only support a narrow field of applications. 
Li \cite{li2019879gops} introduces CNN accelerators in robot hardware architecture, making it possible to support evolution of future algorithms. However, the CNN accelerator in this work is only used for feature-point extraction, it does not use the accelerator for other applications such as place recognition.

\subsection{ Scheduling mode of CNN accelerators }

To accelerate CNN on FPGA, some previous works design frameworks to generate a specific hardware architecture for a target CNN, based on  RTL \cite{li_high_2016} or HLS \cite{lu_evaluating_2017}. These works need to reconfigure the FPGA while running another CNN, and do not support fast switching between different CNNs. Some other works design instruction-driven accelerators \cite{yu2018instruction,qiu2016going}, making rapid switching possible by providing different instruction sequences. However, previous instruction-driven CNN accelerators need to schedule the entire CNN, and can not automatically schedule two or more tasks simultaneously. The inability of CNN accelerators to support multi-task makes it difficult for robotics researchers to use embedded FPGA.